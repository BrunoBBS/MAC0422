\documentclass[12pt,a4paper]{article}
\usepackage[brazil]{babel}
\usepackage[utf8]{inputenc}
\usepackage{graphicx}
\usepackage{times}
\usepackage{url}
\usepackage{algorithm}
\usepackage{algorithmic}
\usepackage[bottom=2cm,top=2cm,left=2cm,right=2cm]{geometry}

\title{Relatório do EP2\\MAC0422 -- Sistemas Operacionais -- 2s2017}
\author{Anderson Andrei da Silva (8944025), Bruno Boaventura Scholl (9793586), Aluno 3 (NUSP)}
\date{}

\begin{document}
\maketitle

\section{Introdução}

\subsection{O problema}

	Uma das varias modalidades de ciclismo realizada em velodromos é a corrida por pontos. O objetivo deste EP sera simular essa modalidade.
Na corrida por pontos, ciclistas iniciam a prova ao mesmo tempo no mesmo lado do velodromo.
	A prova geralmente dura 160 voltas em um velodromo de 250m, no caso da prova masculina, e 100
voltas em um velodromo de 250m, no caso da prova feminina. A pontuação é definida em “sprints”
que acontecem a cada 10 voltas, com 5, 3, 2 e 1 ponto(s) sendo atribuıdos às 4 primeiras colocações em cada sprint. Ao termino da prova, a pontuacão acumulada define as colocações de cada ciclista. Além da pontuacão definida a cada “sprint”, ciclistas que conseguem completar 1 volta sobre todos(as) os(as) outros(as), ganham 20 pontos.
	A simulacão deve considerar que a corrida é em um velódromo com d metros, que n ciclistas
comecam a prova e que v voltas serao realizadas (d > 249, 5 < n <= 5 x de v é múltiplo de 20).
A qualquer momento, no maximo, apenas 10 ciclistas podem estar lado a lado em cada ponto da pista.
Considere que cada ciclista ocupa exatamente 1 metro da pista.


%% Apresentar a organização do EP -- Todos os arquivos presentes no
%% .tar.gz, para que serve cada um deles, etc...
\subsection{Disposição dos arquivos} 
	Os arquivos estão dispostos da seguinte maneira:
    \begin{itemize} %pastas
    
    \item Na \textbf{raiz do diretório} se encontram : Makefile, relatório (em LaTex) e as pastas descritas a seguir;
    \item Na pasta \textbf{execsrc} se encontram dispostos todos os arquivos que gerarão executáveis:
    	
 	\begin{itemize} %arquivos
    \item ep2.c : %%colocar o cabeçalho de cada aquivo aqui
 	\end{itemize} %arquivos
    
    \item Na pasta \textbf{src} se encontram todos os arquivos em c que serão utilizados mas não gerarão executáveis:
    
 	\begin{itemize} %arquivos
    \item rider.c : %%colocar o cabeçalho de cada aquivo aqui
    \item velodrome.c : %%colocar o cabeçalho de cada aquivo aqui
 	\end{itemize} %arquivos
    
    \item Na pasta \textbf{include} se encontram todos os arquivos de include (.h) que serão utilizados nos outros dois casos:
     
    \begin{itemize} %arquivos
    \item ep2.h : %%colocar o cabeçalho de cada aquivo aqui
    \item reider.h : %%colocar o cabeçalho de cada aquivo aqui
 	\item typedef.h : %%colocar o cabeçalho de cada aquivo aqui
 	\item velodrome.h : %%colocar o cabeçalho de cada aquivo aqui
 	\end{itemize} %arquivos
    
        \item Na pasta \textbf{documentation} se encontram este relatŕoio e a apresentação em slides do projeto.
    
    \end{itemize} %pastas
    
%%Apresentar o algoritmo de cada ciclista e o algoritmo geral do código,
%%destacando cada um dos mecanismos utilizados que tenham sido vistos na
%%sala de aula. Cada um desses mecanismos deverá ser mais detalhado nas
%%subseções a seguir (os títulos são exemplos supondo que eles tenham
%%sido usados)
\subsection{Algoritmos}

	Para a resolução do problema, foram criados e utilizados pelo grupo os seguintes algoritmos:
    
\subsubsection{Ciclistas}

\subsubsection{Corrida}

	A seguir serão descritos alguns dos termos usados.
    
\subsection{Barreiras de sincronização}

Explicar para que foram usadas barreiras (em todos os lugares onde
foram usadas), como foram implementadas (se foi implementado algum
algoritmo visto em sala de aula ou se foi usada alguma função pronta).
Caso tenha usado alguma função pronta, expliar o que ela faz por
"baixo dos panos".

\subsection{Semáforos}

Explicar para que foram usados semáforos (em todos os lugares onde
foram usados).

\subsection{Threads}

Explicar para que foram usadas threads (em todos os lugares onde foram
usadas).

\section{Resultados}

Explicar como foram feitos os experimentos (se manualmente ou se
usando alguma linguagem de script ou algo similar. Caso tenha usado
algo automatizado, explicar brevemente como funcionou).

Explicar os resultados (os gráficos que já foram pedidos no
enunciado) deixando claro se foram os esperados ou não. \textbf{Todos}
os gráficos precisam ser citados no texto. Não pode simplesmente
``jogar'' 100 gráficos aqui e falar algo como: ``Em todos os gráficos
pode ser visto...''. Tem que explicar um por um mesmo que a explicação
fique repetitiva. Até pode-se fazer referência a um resultado similar
escrevendo algo como: ``Como pode-se ver no gráfico da Figura 1, o
resultado foi similar ao gráfico da Figura 2 pois ...''.

\section{Divisão das tarefas de implementação}

Mostrar como ficou a dedicação de cada membro da equipe em termos de
implementação. É altamente recomendado que aqui seja apresentada a
saída de algum repositório que já gere o histórico dos
\texttt{commits}.

\section{Referências}

Colocar todas as referências usadas para a escrita do relatório (Se
vocês souberem usar bibtex, podem usar. Se não souberem usar, sem
problemas simplesmente listar as referências aqui).

\end{document}
